\documentclass[]{beamer}
%\usetheme[]{paloalto}
\usetheme[]{}


\usepackage{amsmath}
\usepackage{graphicx}
\usepackage{multimedia}

\setbeamertemplate{footline}[frame number]



\title{
Word space models with a small data set
}
\author{G. Montero~Melis}
\institute{Centre for Research on Bilingualism \\Stockholm University}
\date{11 June 2013}

\begin{document}

\AtBeginSection[]{%
  \begin{frame}<beamer>
    \frametitle{Roadmap}
    \tableofcontents[sectionstyle=show/hide,subsectionstyle=hide/show/hide]
  \end{frame}
  \addtocounter{framenumber}{-1}% If you don't want them to affect the slide number
}


\begin{frame}[fragile]
	\maketitle
\end{frame}


\begin{frame}
	Describe the following event!
\end{frame}

\begin{frame}
	% Check if works!
	\movie[width=15cm,height=10cm,poster,externalviewer]{click here}{pgm_paqtoi_01.avi}
\end{frame}

\begin{frame}
	\frametitle{Overview}
	\tableofcontents
\end{frame}


\section{Background: what the study is about}


\begin{frame}
	\frametitle{An illustration: the event you just saw}
	\begin{itemize}
	\item Simple event, yet a lot of information
	\item Speakers of different languages focus on different aspects
	\item Typical examples in three languages: \pause
	\item English: \emph{He \alert{pushes} a box \alert{up} the roof.} \pause
	\item Swedish: \emph{Han \alert{puttar} ett paket \alert{uppf\"{o}r} ett tak.} 
		\\(equivalent to English)\pause
	\item Spanish: \emph{\alert{Sube} una caja al tejado.}
		\\`He ascends a box to the roof.'
	\end{itemize}
\end{frame}


\begin{frame}
	\frametitle{Summing up: the difference}
	\begin{table}
		\begin{tabular}{r c c c}
		Semantic component	& English/Swedish	& Spanish \\
		\hline
		Path: up			& Y				& Y \\
		Manner in which agent \\moves box (pushing) & Y & N \\
		\end{tabular}
	\caption{Comparison of information expressed across languages}
	\end{table}
	\pause
	\begin{center}
	These systematic differences is what I'm interested in!
	\end{center}
\end{frame}


\begin{frame}
	\frametitle{Main point}
	\begin{itemize}
		\item When we speak about the world we don't just describe it objectively
		\item Talking about something is choosing \emph{what to say} among many possibilities
		\item Different languages guide their speakers towards different aspects of reality
		\item Languages are not equivalent
	\end{itemize}
\end{frame}


%\begin{frame}
%	\frametitle{The language we speak and the way we think}
%	\begin{itemize}
%		\item The language we speak affects the way we think.\pause
%		\item ``We dissect nature along lines laid down by our native language'' (Whorf 1956).\pause
%		\item well, maybe not that radical, but there probably is some type of effect \ldots
%	\end{itemize}
%\end{frame}


\begin{frame}
	\frametitle{Why is this relevant?}
	\begin{itemize}
		\item My field of research: second language acquisition
		\item We all speak in a second language (English right now)
		\item Do we reorganize our semantic/conceptual categories when learning a new language?
		\item Do we express ourselves as in our mother tongue, with the formal make-up of the second language?
	\end{itemize}
\end{frame}



%\begin{frame}
%	\frametitle{}
%	\begin{itemize}
%	
%	\end{itemize}
%\end{frame}


\section{The data}

\begin{frame}
	\frametitle{}
	\begin{itemize}
	\item 
	\end{itemize}
\end{frame}

\section{Some analyses}



\section{Questions}

\begin{frame}
	\frametitle{Validity of the approach}
\end{frame}


\begin{frame}
	\frametitle{Issues related to size of data set}
\end{frame}


\begin{frame}
	\frametitle{}
\end{frame}


\end{document}
