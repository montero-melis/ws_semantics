\documentclass[]{beamer}
%\usetheme[]{paloalto}
\usetheme[]{}


\usepackage{amsmath}
\usepackage{graphicx}
\usepackage{multimedia}

\setbeamertemplate{footline}[frame number]

% To avoid that backup slides are counted as slides during the presentation
\newcommand{\backupbegin}{
   \newcounter{framenumberappendix}
   \setcounter{framenumberappendix}{\value{framenumber}}
}
\newcommand{\backupend}{
   \addtocounter{framenumberappendix}{-\value{framenumber}}
   \addtocounter{framenumber}{\value{framenumberappendix}} 
}

\title{
Word space models with a small data set
}
\author{G. Montero~Melis}
\institute{Centre for Research on Bilingualism \\Stockholm University}
\date{11 June 2013}


\begin{document}

\AtBeginSection[]{%
  \begin{frame}<beamer>
    \frametitle{Roadmap}
    \tableofcontents[sectionstyle=show/hide,subsectionstyle=hide/show/hide]
  \end{frame}
  \addtocounter{framenumber}{-1}% If you don't want them to affect the slide number
}


\begin{frame}[fragile]
	\maketitle
\end{frame}


\begin{frame}
	Describe the following event!
\end{frame}

\begin{frame}
	% Check if works!
	\movie[width=10cm,height=7cm,poster,externalviewer]{Embedd video here}{pgm_paqtoi_01.avi}
\end{frame}

\begin{frame}
	\frametitle{Overview}
	\tableofcontents
\end{frame}


\section{Background: what the study is about}


\begin{frame}
	\frametitle{An illustration: the event you just saw}
	\begin{itemize}
	\item Simple motion event, yet a lot of information
	\item Speakers of different languages will focus on different components of this event when describing it
	\item Typical examples in three languages: \pause
	\item English: \emph{He \alert{pushes} a box \alert{up} the roof.} \pause
	\item Swedish: \emph{Han \alert{puttar} ett paket \alert{uppf\"{o}r} ett tak.} 
		\\(equivalent to English)\pause
	\item Spanish: \emph{\alert{Sube} una caja al tejado.}
		\\`He ascends a box to the roof.'\pause
	 \item Of course there is a great amount of variability
	\end{itemize}
\end{frame}


\begin{frame}
	\frametitle{Summing up: the difference}
	\begin{table}
		\begin{tabular}{r c c c}
		Semantic component	& English/Swedish	& Spanish \\
		\hline
		Path (\emph{up})			& Y				& Y \\
		Manner in which agent \\moves box (\emph{pushing}) & Y & \alert{N} \\
		\end{tabular}
	\caption{Comparison of information expressed across languages}
	\end{table}
	\pause
	\begin{center}
	This kind of systematic differences is what I'm interested in!
	\end{center}
\end{frame}


\begin{frame}
	\frametitle{Main point}
	\begin{itemize}
		\item When we speak about the world we don't just describe it objectively
		\item Talking about something is choosing \emph{what to say} among many possibilities
		\item Different languages guide their speakers towards expressing different aspects of reality
		\item Languages are not fully equivalent
	\end{itemize}
\end{frame}


%\begin{frame}
%	\frametitle{The language we speak and the way we think}
%	\begin{itemize}
%		\item The language we speak affects the way we think.\pause
%		\item ``We dissect nature along lines laid down by our native language'' (Whorf 1956).\pause
%		\item well, maybe not that radical, but there probably is some type of effect \ldots
%	\end{itemize}
%\end{frame}


\begin{frame}
	\frametitle{Why is this relevant?}
	\begin{itemize}
		\item My field of research: second language acquisition
		\item We all speak in a second language (English right now)
		\item Do we reorganize our semantic/conceptual categories when learning a new language?
		\item Do we express ourselves as in our mother tongue, with the formal make-up of the second language?
	\end{itemize}
\end{frame}



\begin{frame}
	\frametitle{The study (at least part of it\ldots)}
	\begin{description}
		\item[Research question] Do native speakers of Swedish pick out different semantic components than native speakers of Spanish when describing events?\pause
		\item[Case in point] Caused motion events, in which an agent causes an object to move, e.g. `He pulls a chair into the cave.' 
		\item[Participants] Approx. 80 participants (40 per language)
		\item[Materials] 32 target videoclips, 7 distractor items, one training item
		\item[Procedure] Participants see each videoclip and describe it orally
%		\item[Prompt] `What happened? Focus on the action.'
	\end{description}
\end{frame}


\section{Approach and data}

\begin{frame}
	\frametitle{Vector space model}
	\begin{itemize}
		\item Each videoclip is indexed by a participant's description
		\item Average description length: 10 words
		\item Each description is represented as a vector containing information about the words used in the description
		\item The $m$ distinct words and the $n$ videoclips are represented in a term-by-videoclip matrix
		\item In practice, roughly a $1000 \times 40$ matrix (unique words$\times$videoclips)
	\end{itemize}
\end{frame}


\begin{frame}
	\frametitle{Term-by-videoclip matrix for one participant}
	\begin{table}[ht]
		\begin{center}
		\begin{tabular}{rrrrr}
		  \hline
		 & dis\_avarou & pge\_tabgro & pgt\_panrue & prt\_rourue \\ 
		  \hline
		\aa ker &   1 &   0 &   0 &   0 \\ 
		  \"{o}ver &   0 &   0 &   1 &   1 \\ 
		  berget &   0 &   0 &   0 &   0 \\ 
		  bl\aa &   1 &   0 &   0 &   0 \\ 
		  bollen &   2 &   0 &   0 &   0 \\ 
		  bord &   0 &   1 &   0 &   0 \\ 
		  drar &   0 &   0 &   1 &   0 \\ 
		  en &   0 &   0 &   1 &   0 \\ 
		  ett &   0 &   1 &   0 &   1 \\ 
		  frukt &   0 &   0 &   1 &   0 \\ 
		  gatan &   0 &   0 &   1 &   1 \\ 
		  grottan &   0 &   1 &   0 &   0 \\ 
		 \hline
		\end{tabular}
		\end{center}
		\caption{Sample data from term-by-videoclip matrix (one participant)}
	\end{table}
\end{frame}



\begin{frame}
	\frametitle{Parallel texts}
	\begin{itemize}
		\item Aligned parallel data: the same document (i.e. document with the same content) is available in several languages \pause
		\item Here, each description of a videoclip is available in Swedish and Spanish 
	 	\item Actually, for each videoclip there are 40 Swedish and 40 Spanish descriptions
	 	\item Data can be grouped by language to yield two term-by-videoclip matrices
	 	\item These represent ``average'' descriptions for each videoclip in each language
	 	\item NB: These two aggregated matrices are still \emph{very} sparse!
	\end{itemize}
\end{frame}

\begin{frame}
	\frametitle{Term-by-videoclip matrix for all Swedish participants}
	\begin{table}[ht]
		\begin{center}
		\begin{tabular}{rrrrr}
		  \hline
		 & tgm\_sactoi & tgt\_cherue & pge\_tabgro & tge\_chamai \\ 
		  \hline
		iv\"{a}g &   0 &   0 &   0 &   0 \\ 
		  lilla &   0 &   0 &   0 &   0 \\ 
		  flyttar &   0 &   0 &   1 &   0 \\ 
		 upp &   4 &   8 &   0 &  12 \\ 
		  gul &   0 &   0 &   0 &   0 \\ 
		  bok &   0 &   0 &   0 &   0 \\ 
		  stolen &   0 &   0 &   0 &   3 \\ 
		   ett &   0 &   1 &   3 &   0 \\ 
		  krockade &   2 &   0 &   0 &   0 \\ 
		  vid &   0 &   0 &   0 &   0 \\ 
		   \hline
		\end{tabular}
		\end{center}
		\caption{Sample data from term-by-videoclip matrix (all Swedish participants)}
	\end{table}
\end{frame}




\section{Some analyses}


\begin{frame}
	\frametitle{The basic question}
	\begin{center}
	Do events cluster differently in the two languages?
	\end{center}
	\pause
	\begin{itemize}
	\item Point of departure for analysis is the term-by-videoclip matrix
	\item However, this matrix can be analyzed in different ways
	\item I draw my tools from latent semantic analysis (LSA)\ldots
	\end{itemize}
\end{frame}


\begin{frame}
	\frametitle{Latent semantic analysis (LSA)}
	\begin{itemize}
		\item Theory of meaning based on statistical computations of text data (e.g. Deerwester et al. 1990; Landauer 2007)
		\item LSA ``uncovers the underlying or `latent' semantic structure'' in the data by removing noise (Martin \& Berry 2007: 36)
		\item It handles two important sources of noise: synonymy (two words with same meaning) and polysemy (a word with more than one meaning)
		\item In practice, LSA creates a \textbf{semantic space}, where both terms and documents co-exist \pause
		\item Mathematically based on a truncated Singular Value Decomposition or two-way factor analysis
		\item Computationally expensive, but computational issues are not a big concern here
		\item R-package `lsa'
	\end{itemize}
\end{frame}

\begin{frame}
\begin{center}
Some quick-and-dirty exploratory analyses\ldots
\end{center}
\end{frame}

\begin{frame}
	\frametitle{Closest neighbours to \emph{sube} (Sp., `ascends')}
	\begin{table}[ht]
\begin{center}
\begin{tabular}{rr}
  \hline
 & cosine similarity \\ 
  \hline
uppf\"{o}r & 0.97 \\ 
  upp & 0.95 \\ 
  subido & 0.94 \\ 
  subiendo & 0.88 \\ 
  arriba & 0.86 \\ 
  asciende & 0.83 \\ 
  encima & 0.82 \\ 
  hustak & 0.82 \\ 
     \hline
\end{tabular}
\end{center}
\caption{Closest neighbours to \emph{sube} (Sp., `ascends'), threshold $0.8$}
\end{table}
\begin{itemize}
	 \item Based on all data collapsed for each document
	 \item I.e. Here each document contains all the words used by all participants to describe that event
\end{itemize}
\end{frame}


\begin{frame}
\frametitle{Closest neighbours to \emph{drog} (Sw., `pulled', `drew')}
\begin{table}[ht]
\begin{center}
\begin{tabular}{rr}
  \hline
 & cosine similarity \\ 
  \hline
drar & 0.97 \\ 
  efter & 0.90 \\ 
  arrastrando & 0.89 \\ 
  tirando & 0.87 \\ 
  arrastra & 0.85 \\ 
   \hline
\end{tabular}
\end{center}
\caption{Closest neighbours to \emph{drog} (Sw., `pulled', `drew'), threshold $0.8$}
\end{table}
\end{frame}


\begin{frame}
	\frametitle{Multidimensional scaling (MDS) of similarity among videoclips, grouped by language}
	\begin{figure}
		\centering
		\includegraphics[width=.65\textwidth]{MDS}
		\caption{Plots of classical MDS in two dimensions of the distance between documents as measured by cosine similarity}
	\end{figure}
	\begin{itemize}
	\item First (horizontal) dimension clearly distinguishes distractor items (left) from target items (right)
	\end{itemize}
\end{frame}


\begin{frame}
	\frametitle{Sum-up}
	 \begin{itemize}
	 	\item A series of analyses are interpretable: indicates that the approach is sensible
	 	\item Great amount of semantic overlap between both languages
	 	\item Some differences are found but it remains difficult to evaluate their significance, both statistical and `true' significance
	 \end{itemize}
\end{frame}



\section{Questions}

\begin{frame}
	\frametitle{Validity of the approach}
	\begin{itemize}
	\item People who use these methods normally do not pose this kind of questions
	\item People who ask this kind of questions generally do not use these methods
	\item Do you see any flagrant pitfall in this approach
	\end{itemize}
\end{frame}


\begin{frame}
	\frametitle{Issues related to size of data set}
	\begin{itemize}
	\item Word space models like LSA are usually implemented with huge corpora, ``representative of language''
	\item My data set is small but highly structured -- it contains exactly the information I'm interested in
	\item Is this a trade-off that can work?
	\end{itemize}
\end{frame}


\begin{frame}
	\frametitle{Significance issues}
	\begin{itemize}
		\item How to assess statistical significance?
		\item How to judge whether differences are noteworthy or not?
		\item Bound to be exploratory or could there be hypothesis testing (cf. Sverker Sikstr\"{o}m)
	\end{itemize}
\end{frame}


\begin{frame}
	\frametitle{Additional coding?}
	\begin{itemize}
		\item Avoiding interfering with the data is what motivates this approach in the first place, but \ldots
		\item Should data coding be applied, e.g. coding for semantic categories such as `path'  component (e.g., \emph{across}) or `manner' component (e.g., \emph{pulls})
		\item On what ground should these decisions be made?
	\end{itemize}
\end{frame}


\begin{frame}
	\frametitle{Technical issues}
	\begin{itemize}
	\item Parameter setting for LSA or similar
	\item Choice of number of dimensions
	\item Stop lists
	\item Weighting schemes
	\end{itemize}
\end{frame}




\appendix
\backupbegin

\begin{frame}
	\frametitle{}
\end{frame}

\begin{frame}
	\frametitle{More about the data set}
	\begin{table}[ht]
		\begin{center}
		\begin{tabular}{rr}
		  \hline
		  Vocabulary & 968\\
		  Documents & 39 \\
		  frequencies not zero & 4024 \\
		  zero frequencies & 33728\\
		  max term length & 14\\
		  non-alphanumerics in terms &  0 \\
		   \hline
		   \hline
		\end{tabular}
		\end{center}
		\caption{Information about the term-by-videoclip matrix of all participants (only half of participants)}
	\end{table}
\end{frame}


\begin{frame}
	\frametitle{Multidimensional scaling (MDS) of similarity among videoclips, grouped by language (targets only)}
	\begin{figure}
		\centering
		\includegraphics[width=.65\textwidth]{MDS_targets}
		\caption{Plots of classical MDS in two dimensions of the distance between documents as measured by cosine similarity; target items only}
	\end{figure}
	\begin{itemize}
	\item Differences between languages are not clear at first sight.
	\end{itemize}
\end{frame}


\begin{frame}
	\frametitle{Hierarchical cluster analysis of Spanish data (Ward's method, two main clusters)}
	\begin{figure}
	\includegraphics[width=.75\textwidth]{hclust_spa_k2}
	\end{figure}
	\begin{itemize}
	\item Events of going \textbf{down} (left cluster) vs. rest of directions (right cluster)
	\end{itemize}
\end{frame}

\begin{frame}
		\frametitle{Hierarchical cluster analysis of Swedish data (Ward's method, two main clusters)}
		\begin{figure}
	\includegraphics[width=.75\textwidth]{hclust_swe_k2}
	\end{figure}
	\begin{itemize}
	\item Events of going \textbf{across} (left cluster) vs. rest of directions (right cluster)
	\end{itemize}
\end{frame}

\begin{frame}
	\frametitle{Hierarchical cluster analysis of Spanish data (Ward's method, seven main clusters)}
	\begin{figure}
	\includegraphics[width=\textwidth]{hclust_spa_k7}
	\end{figure}
\end{frame}

\begin{frame}
		\frametitle{Hierarchical cluster analysis of Swedish data (Ward's method, seven main clusters)}
		\begin{figure}
	\includegraphics[width=\textwidth]{hclust_swe_k7}
	\end{figure}
\end{frame}


\begin{frame}[fragile]
	\frametitle{Example of aggregated document, all Swedish descriptions included (pgm\_paqtoi)}
	\begin{verbatim}
	Hopi/skjuter/upp/ett/paket/upp/på/ett/tak/puttade/
	ett/paket/upp/på/ett/tak/Hopi/puttar/ett/paket/uppför/
	ett/sluttande/tak/han/puttar/upp/ett/stort/paket/
	uppför/taket/han/drar/upp/ett/paket/för/ett/tak/han/
	gick/upp/med/ett/stort/paket/eh/för/ett/hustak/han/
	puttar/ett/eh/paket/uppför/ett/tak/Hopi/eh/skjuter/
	paketet/uppför/husets/tak/han/skjuter/ett/stort/
	paket/uppför/hustaket/nu/puttade/han/upp/ett/paket/uppför/
	taket/eh/Hopi/sköt/paketet/uppför/taket/och/nu/skjuter/
	han/ett/paket/uppför/en/husvägg/han/sköt/upp/ett/paket/
	på/ett/tak/eh/han/puttade/upp/ett/eh/lila/paket/för/samma/
	hustak/som/han/tidigare/puttade/upp/en/badring/på/eh/
	Hopi/skjuter/skjuter/upp/ett/paket/eh/uppför/ett/
	lutande/tak/mm/Hopi/sköt/ett/eh/lila/paket/upp/till/
	eh/taknocken/på/ett/hus/
	\end{verbatim}
\end{frame}


\backupend

\end{document}
